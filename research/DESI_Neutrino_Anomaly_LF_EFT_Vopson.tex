\documentclass[10pt,a4paper]{article}
\usepackage[utf8]{inputenc}
\usepackage{amsmath,amssymb}
\usepackage{geometry}
\usepackage{hyperref}
\usepackage{cite}

\geometry{top=0.85in,bottom=0.85in,left=0.9in,right=0.9in}
\setlength{\parindent}{0pt}
\setlength{\parskip}{0.35em}

\title{\textbf{On the DESI Neutrino Anomaly: A Chameleon Information Potential Explanation}\\
\large\textit{A conservative LF--EFT note bridging screened gravity and ``infodynamics''}}
\author{\textbf{Jean Charbonneau}\\ \textit{The Light Framework}}
\date{December 28, 2025}

\begin{document}
\maketitle
\vspace{-0.8em}

\textbf{Scope (one sentence).} This note proposes a \emph{minimal, falsifiable} interpretation of the DESI DR2 ``negative neutrino mass'' preference as a \emph{projection artifact} of fitting uncoupled $\Lambda$CDM to data generated by an interacting, screened scalar sector that can be read (optionally) as an ``information potential'' in the spirit of Vopson's information-mass conjectures \cite{Vopson2019,Vopson2022}.

\section*{1.\ \ The observation}
Two empirical pressure points appear in DESI DR2 + CMB combinations:

\textbf{(i) ``Negative'' effective neutrino mass parameter.} In the DESI DR2 neutrino analysis, an \emph{effective} parameter $\sum m_{\nu,\mathrm{eff}}$ is introduced that is allowed to take negative values to remove prior-boundary effects at $\sum m_\nu\ge 0$ and diagnose model mismatch; this extended fit yields a preference in the negative direction and a $3\sigma$-level tension with oscillation lower limits \cite{DESI_Neutrino}. Importantly, this does \emph{not} imply physically negative neutrino rest mass.

\textbf{(ii) Evolving dark energy.} DESI DR2 BAO combined with CMB prefers time-varying $w(z)$ (often fit as $w_0w_a$), with a favored solution in the quadrant $w_0>-1,\ w_a<0$ and improved fit significance versus $\Lambda$CDM \cite{DESI_Cosmology}. In such reconstructions, $w(z)$ can appear ``phantom-like'' ($w<-1$) over part of cosmic history, depending on the dataset combination and parameterization.

\section*{2.\ \ Why $\Lambda$CDM strains}
In standard $\Lambda$CDM, neutrinos (with $\sum m_\nu>0$) suppress growth via free streaming and contribute positively to the background density. If the true universe has additional physics that changes the \emph{growth--distance} relationship (e.g.\ an extra long-range interaction acting primarily in low-density environments), then a constrained $\Lambda$CDM fit may push nuisance/degenerate directions into unphysical territory \emph{when permitted} (e.g.\ $\sum m_{\nu,\mathrm{eff}}<0$) as a diagnostic of model mis-specification rather than a literal negative mass \cite{DESI_Neutrino}.

\section*{3.\ \ LF--EFT mechanism: screened scalar coupled to the diffuse sector}
We consider a conservative screened scalar--tensor EFT (``chameleon class'') with an optional information-theory interpretation:

\begin{equation}
S=\!\int d^4x\sqrt{-g}\left[\frac{M_{\rm Pl}^2}{2}R-\frac12(\partial\phi)^2 - V(\phi)\right]
+S_m[\psi_m,A_m^2(\phi)g_{\mu\nu}]+S_\nu[\psi_\nu,A_\nu^2(\phi)g_{\mu\nu}],
\end{equation}
with
\begin{equation}
A_i(\phi)=e^{\beta_i\phi/M_{\rm Pl}}\simeq 1+\beta_i\phi/M_{\rm Pl},\qquad
V(\phi)=\Lambda^{4+n}\phi^{-n}\ (n>0),
\end{equation}
a standard chameleon template \cite{KhouryWeltman2004}. The environment-dependent effective potential
\begin{equation}
V_{\rm eff}(\phi)=V(\phi)+\sum_i \rho_i A_i(\phi)
\end{equation}
yields density-dependent $\phi_{\min}(\rho)$ and $m_{\rm eff}(\rho)$; in low-density regions (cosmic voids) $\phi$ can become light/unscreened, enhancing a fifth force, while remaining screened in dense regions (Solar System, laboratory) \cite{KhouryWeltman2004}.

\textbf{Neutrino-specific lever.} Allow $\beta_\nu\neq \beta_m$ (an \emph{assumption} to be tested): the interaction can be arranged to act most strongly on the ``diffuse'' component. The coupling is proportional to the trace $T=\rho-3p$; thus the effect is automatically suppressed when neutrinos are ultra-relativistic ($T_\nu\simeq 0$) and turns on when they become non-relativistic ($T_\nu\simeq -\rho_\nu$), i.e.\ at late times relevant to BAO-era observations.

\textbf{Infodynamics bridge (interpretive, not required).} If one adopts Vopson's conjecture that information carries mass/energy \cite{Vopson2019} and his ``infodynamics'' proposal \cite{Vopson2022}, then $\phi$ may be interpreted as an \emph{information potential} whose screening implements an effective drive toward higher information density (lower information entropy) while respecting ordinary thermodynamic constraints. The EFT above remains the testable layer.

\section*{4.\ \ Why it can look like ``negative mass'' in $\Lambda$CDM fits}
If neutrinos exchange energy--momentum with $\phi$, they are not separately conserved:
\begin{equation}
\nabla_\mu T^{\mu\nu}_{(\nu)}=+Q^\nu,\qquad \nabla_\mu T^{\mu\nu}_{(\phi)}=-Q^\nu,\qquad
\nabla_\mu(T^{\mu\nu}_{(\nu)}+T^{\mu\nu}_{(\phi)})=0,
\end{equation}
with (schematically) $Q\propto \beta_\nu(\rho_\nu-3p_\nu)\nabla^\nu\phi/M_{\rm Pl}$ in conformal couplings. A $\Lambda$CDM analysis assumes $Q=0$ and will then project the true interacting evolution into \emph{effective} parameters. One such projection is precisely the DESI $\sum m_{\nu,\mathrm{eff}}$ extension, which allows negative values to absorb mis-modeling of the expansion/growth response \cite{DESI_Neutrino}. In LF--EFT language: the neutrinos remain physical, but an unscreened $\phi$ in low-density regions enhances clustering/attraction and can counteract the standard ``neutrino suppression'' signature, biasing an uncoupled fit toward an unphysical negative effective neutrino contribution.

\section*{5.\ \ Falsifiable predictions and immediate tests}
\textbf{(A) Fit replacement test (most direct).} Refit DESI DR2 + CMB in an interacting $(\phi,\nu)$ model with screening constraints enforced (Solar System / lab bounds). Prediction: the posterior for physical $\sum m_\nu$ returns to $\ge 0$ while maintaining (or improving) the likelihood gain that, in $\Lambda$CDM extensions, shows up as $\sum m_{\nu,\mathrm{eff}}<0$ and/or evolving $w(z)$ \cite{DESI_Cosmology,DESI_Neutrino}.

\textbf{(B) Environment dependence.} Because screening depends on density, deviations should correlate with void statistics / low-density volumes (e.g.\ void lensing, scale-dependent growth in underdensities), not with high-density systems.

\textbf{(C) Coupling signature.} The effect should scale with $(\rho_\nu-3p_\nu)$, predicting negligible impact when neutrinos are relativistic and increasing influence as they become non-relativistic, providing a time/scale signature distinguishable from simply changing $\sum m_\nu$.

\section*{Conclusion}
The DESI ``negative $\sum m_{\nu,\mathrm{eff}}$'' preference can be read as a \emph{diagnostic} of missing physics rather than an error or a literal negative neutrino mass. A screened scalar EFT with a neutrino-sensitive coupling provides a conservative, testable mechanism: in low-density/high-entropy regimes the field unscreens and modifies growth in a way that an uncoupled $\Lambda$CDM fit may misinterpret as a negative effective neutrino contribution and/or phantom-like $w(z)$. If Vopson's information-mass/infodynamics conjectures are adopted, $\phi$ can be interpreted as an information potential: the universe's large-scale dynamics resembles an optimization flow in information density \cite{Vopson2019,Vopson2022}---but the \emph{physics claim} remains falsifiable through the fit-replacement and environment-dependence tests above.

\vspace{-0.25em}
\begin{thebibliography}{9}

\bibitem{DESI_Cosmology}
DESI Collaboration, ``DESI DR2 Results II: Measurements of Baryon Acoustic Oscillations and Cosmological Constraints,'' arXiv:2503.14738 (2025). \url{https://arxiv.org/abs/2503.14738}

\bibitem{DESI_Neutrino}
DESI Collaboration, ``Constraints on Neutrino Physics from DESI DR2 BAO and DR1 Full Shape,'' arXiv:2503.14744 (2025). \url{https://arxiv.org/abs/2503.14744}

\bibitem{KhouryWeltman2004}
J.~Khoury and A.~Weltman, ``Chameleon Fields: Awaiting Surprises for Tests of Gravity in Space,'' Phys.\ Rev.\ Lett.\ 93, 171104 (2004). \url{https://arxiv.org/abs/astro-ph/0309300}

\bibitem{Vopson2019}
M.~M.~Vopson, ``The mass-energy-information equivalence principle,'' AIP Advances 9, 085206 (2019). \url{https://ui.adsabs.harvard.edu/abs/2019AIPA....9i5206V/abstract}

\bibitem{Vopson2022}
M.~M.~Vopson, ``Second law of information dynamics,'' AIP Advances 12, 075310 (2022). \url{https://pubs.aip.org/aip/adv/article/12/7/075310/2819368/Second-law-of-information-dynamics}

\end{thebibliography}

\end{document}
